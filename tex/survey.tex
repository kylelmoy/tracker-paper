\documentclass[main.tex]{subfiles}
\begin{document}
\section{Background}
 
\subsection{Worm Tracking}
There already many worm trackers available on the market. Many research groups have published their hardware specifications and code. 

Here are the main players:
\begin{table}[!htbp]
  \centering
  \begin{tabular}{l l l l p{5cm}}
\toprule
Lab          & Ref                   & Tracker name          & Goal              & Institution                                               \\
\midrule
Schafer      & \cite{yemini2013}     & WormTracker 2.0       & Phenotyping       & UCSD, Oxford*                                             \\
Goodman      & \cite{ramot2008}**    & Parallel Worm Tracker & Electrophysiology & Stanford                                                  \\
Tavernarakis & \cite{tsibidis2007}** & Nemo                  & Neurodegeneration & Institute of Molecular Biology and Biotechnology (Greece) \\
Hoshi        & \cite{hoshi2006}      &                       &                   & Iwate University (Japan)                                  \\
Lockery      & \cite{pierce1999}     &                       &                   & University of Oregon                                      \\
\midrule
\multicolumn{4}{l}{*Previous, Current institutions}                                                                                          \\
\multicolumn{4}{l}{**Post-processing tools, not motorized trackers}                                                                          \\
\bottomrule
  \end{tabular}
  \caption{Other research labs with trackers}
  \label{tab:juggernauts}
\end{table}

\subsubsection{Optogenetic imaging studies}

\subsubsection{Imaging}
\begin{table}[!htbp]
  \centering
  \begin{tabular}{lllcll}
\toprule
Ref                 & lab          & fps   & resolution & $\mu$m/px & duration                        \\
\midrule
Medix               & Kim          & 20-45 & 1280 x 960 & 7.8       & 45 min                          \\
\cite{yemini2013}   & Schafer      & 20-30 & 640 x 480  & 3.5 - 4.5 & 15 min                          \\
\cite{ramot2008}    & Goodman      & 7.5   & 640 x 480  &           & 30 sec every 10 min for 190 min \\
\cite{tsibidis2007} & Tavernarakis &       &            &           &                                 \\
\cite{hoshi2006}    & Hoshi        & 3     & 640 x 512  & 20x mag   & 5 and 70 min                    \\
\cite{geng2004}     & Schafer      &       &            & 50x mag   &                                 \\
\cite{pierce1999}   & Lockery      & 1     & 608 x 480  & 6.172     & 10 min                          \\ 
\bottomrule
  \end{tabular}
  \caption{Imaging capabilities of comparable worm trackers}
  \label{tab:trackers-img}
\end{table}


In \cite{pierce1999}, , they used a 9 cm plate. The plate was on the motorized stage. Frames dropped during motion made for a variable sampling rate. 
In \cite{hoshi2006}, the plate was on the X-Y stage. They used a 15cm plate without fooooooood!!! Drop frames during X-Y stage movement. two experiments: one for 5 min with many mutants (22) and many animals. One for 70 min using wild type and 2 mutants. 

In \cite{tsibidis2007}

In \cite{ramot2008}, the word tracking is overloaded. Their software Parallel Worm Tracker identifies individual worms in multi-worm dish and records the track generated by following the centroid accros frames. Their tracker performs the feature extraction workload. The worm's centroid, and information about the size and shape are recorded during tracking. This paper is misleading. There is actually no tracking occuring to move a camera or dish. Tracking means something else in this paper. They are using grayscale!
 
\subsubsection{Video Recording}
\subsubsection{Stage/Camera Motion}

\subsection{Representing the worm for analysis}
\subsection{Feature Extraction}
\begin{table}[!htbp]
  \centering
  \begin{tabular}{ccccc}
\toprule
Ref     & research group &  & resolution & pixel spacing \\
\midrule
\cite{} &                                                \\

\bottomrule
  \end{tabular}
  \caption{Imaging capabilities of comparable worm trackers}
  \label{tab:features}
\end{table}

\subsection{Software}
\begin{table}[!htbp]
  \centering
  \begin{tabular}{cccccc}
\toprule
Ref                 & lab          & Language         \\
\midrule
\cite{yemini2013}   & Schafer      &                  \\
\cite{ramot2008}    & Goodman      &                  \\
\cite{tsibidis2007} & Tavernarakis &                  \\
\cite{hoshi2006}    & Hoshi        & Visual Basic 6.0 \\
\cite{pierce1999}   & Lockery      &                  \\
\bottomrule
  \end{tabular}
  \caption{Programming language preferences}
  \label{tab:trackers-lang}
\end{table}


\end{document}
